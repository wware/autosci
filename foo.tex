\documentclass{article}
\usepackage[colorlinks=true,urlcolor=blue,linkcolor=blue]{hyperref}
\title{Semantic markup of \LaTeX and Markdown documents}
\author{Will Ware}
\date{24 February 2016}
\begin{document}
\maketitle
\section {Preliminaries}
My goal here is to figure out what I call a \textit{machine-tractable} scientific literature:
a nice clean way to embed machine-processable semantic information in the source of LaTeX
and Markdown documents, such software pre-processors can extract the information in
machine-tractable form and can also generate (ideally publication-ready) HTML or PDF output.

This idea is part of a larger
\href{http://willware.blogspot.com/2013/10/bar-camp-boston-2013-talk-on-automation.html}{group of ideas}
that I have been thinking about since around 2010.

\pretolerance=150
\begin{quotation}
...to formulate a linked language of science that machines can understand. Publish
papers in formats like RDF/Turtle or JSON or JSON-LD or YAML. Link scientific
literature to existing semantic networks (DBpedia, Freebase, Google Knowledge Graph,
LinkedData.org, Schema.org etc). Create schemas for scientific domains and for the
scientific method (hypotheses, predictions, experiments, data). Provide tutorials,
tools and incentives to encourage researchers to publish machine-tractable papers.
Create a distributed graph or database of these papers, in the role of scientific
journals, accessible to people and machines everywhere. Maybe use Stackoverflow as
a model for peer review.
\end{quotation}

My immediate concern is with the task of comfortably embedding such formats in LaTeX or
Markdown source. I say \textit{comfortably} because I hope for eventual wide adoption by
the authors of scientific literature, and that is unlikely if semantic markup presents any
significant additional burden to publication. Consequently the syntax for such markup must
be simple and its meaning obvious.

In particular it should not be necessary to specify the same idea twice, once in English
and again in machine-tractable form. As with mathematical equations, the machine-tractable
representation should be readable and expressive to the human audience as well as the
machine.

\subsection{Machine representations}

There have been huge advances in machine learning in recent years, fueled by large
investments from Google, Facebook, and other organizations. I have not kept up with
this work as well as I would like, and no doubt my ideas in this area will seem
antiquated to those who have.

I have tried not to commit this approach to any particular representation, but I've
tended to use [RDF/Turtle](https://www.w3.org/TR/turtle/) because first-order logic
doesn't sound like a bad place to start for machine reasoning about science. The notion
of specifying information or knowledge as a semantic network seems to me a useful one.

\subsection{Notation}

These considerations led me to adopt a
\href{https://en.wikipedia.org/wiki/Literate_programming}{literate programming} approach,
borrowing notationally from Norman Ramsey's \href{http://www.cs.tufts.edu/~nr/noweb/}{noweb}.
I am \href{https://github.com/JonathanAquino/noweb.py}{not the first} to rewrite noweb in
Python.

In literate programming, one writes a program in small pieces, discussing each in turn.
Likewise, a semantic network can be specified in pieces, with discussion. Here the first
line gives a name to this piece so that it can be referenced elsewhere. The remaining
lines are in the RDF/Turtle language. They tell us that, in the world of Marvel comics
superheros, Spiderman is a person named \textit{Spiderman} who has an enemy (the Green Goblin,
to be discussed shortly).

<<semantic info about spiderman>> =
<#spiderman>
    a foaf:Person ;
    foaf:name "Spiderman" ;
    rel:enemyOf <#green-goblin> .
@

Terms like \textit{foaf:Person} refer to \href{http://xmlns.com/foaf/spec/}{FOAF}, or
\textit{friend of a friend}, a library of semantic relationships between people. Another
similar library is \href{http://vocab.org/relationship/}{Relationship}, which include the
\textit{rel:enemyOf} term.

The components of a semantic network are triplets. Blah blah blah triplets....
semicolons... commans... period.

The Green Goblin is another character in the Marvel comics universe, not so different
from Spiderman in some ways.

<<semantic info about the green goblin>> =
<#green-goblin>
    a foaf:Person ;
    foaf:name "Green Goblin" ;
    rel:enemyOf <#spiderman> .
@

Finally we have a wrapper for the two pieces of semantic network above. Here we
provide referents for the prefixes \textit{foaf} and \textit{rel}, and other prefixes
that are commonly used in RDF.

<<rdf>> =
@base <http://example.org/> .
@prefix rdf: <http://www.w3.org/1999/02/22-rdf-syntax-ns#> .
@prefix rdfs: <http://www.w3.org/2000/01/rdf-schema#> .
@prefix foaf: <http://xmlns.com/foaf/0.1/> .
@prefix rel: <http://vocab.org/relationship/> .
<<semantic info about spiderman>>
<<semantic info about the green goblin>>
@

\section{The bigger picture}

\subsection{Inference}

In RDF terms, \textit{inference} can be considered a synonym for \textit{deduction}.
Mechanically, it's a process whereby some old RDF triples go in, and some new RDF triples
come out. In practice this is done with software called a \textit{reasoner}, which is
given a set of rules defining what constitutes valid inference.

Google \textit{semantic reasoner} or \textit{RDF reasoner} to learn more about reasoners.

\href{https://en.wikipedia.org/wiki/Rule_of_inference}{Inference rules} are often written
using languages like
\href{https://www.w3.org/2001/sw/wiki/OWL}{OWL} which capture some way of thinking
about knowledge. For instance, OWL includes a lot of operators from set theory.

\subsection{Closures}

By \textit{closure} I mean
\href{https://en.wikipedia.org/wiki/Epistemic_closure}{epistemic closure},
not the
\href{https://en.wikipedia.org/wiki/Closure_(computer_programming)}{computer science}
kind. A subset of epistemic closure is
\href{https://en.wikipedia.org/wiki/Closure_(mathematics)}{mathematical closure}.

An example of mathematical closure is the fact that integers are closed under addition.
You can add any two integers and you'll always get an integer result.

An epistemic closure is basically a complete system of thought: given a set of axioms,
you've done all the reasoning to arrive at all possible conclusions, and that whole set
of axioms and conclusions is your state of knowledge.

An \href{https://www.w3.org/TR/2002/WD-rdf-mt-20020214/#rdf_entail}{RDF closure} is
the RDF subset of that, a complete set of mutually inferrable RDF triples. Any triple you
can arrive at by performing inference on the existing triples is another existing triple.
There is no further inference left to be done.

An RDF closure exists in the context of a set of axioms and inference rules.

In this attempt to trivialize science to what is representable in RDF, a state of scientific
knowledge is represented by an RDF closure, which is in turn determined by its axioms and
inference rules. It is may never be necessary to enumerate the entire contents of the
closure.

\subsection{Hypotheses, and Karl Popper}

Assuming you've got a RDF closure representing the state of your scientific knowledge,
the next step is to propose a new hypothesis, which means adding one or more new axioms.
Then you infer whatever you can to arrive at the consequences of your hypothesis. Hopefully
some of these will be empirically testable, and in that event you design experiments and
run them to see if any of the predictions are empirically falsifiable.

If a prediction is falsified then that hypothesis is falsified, at least in the context
of the existing set of hypotheses. Then you have a whole epistemological can of worms,
because maybe it's only incompatible with one previous hypothesis, and maybe that one
previous hypothesis might not be true. Because at no point are hypotheses \textit{proven}
to be true, they have only not yet been falsified.

Assuming for sake of argument that the list of inference rules is essentially constant...

The current set of axioms/hypotheses could be kept in a git repository, and when you
want to test a new hypothesis you create a branch in the repository. That branch doesn't
get merged into master until you've got a high level of confidence in the new hypothesis.

Try to come up with some meaningful numerical measure of confidence in the new hypothesis
and include metadata for that in the branch. If the master branch advances significantly
while the new hypothesis is under study, it would make sense to merge master back into
the hypothesis branch.

\subsection{Designing experiments}

Initially, it probably makes sense for this machine to periodically consult a human and
say

\begin{quotation}
Here are a bunch of hypotheses I've generated, starting from where the master branch
was at time T. For each hypothesis, I've listed several predictions. You, the human,
should decide which of these to pursue, and you should test the predictions
empirically because I don't yet know how to design and run experiments. Please let me
know which hypotheses are non-starters, and which predictions are a-priori known to fail.
\end{quotation}

Over time, the human will probably see opportunities for automation in (a) pruning unproductive
hypotheses and predictions, and (b) turning predictions into experiments. As that happens,
the machine can handle progressively larger portions of the process.

Ross King's \href{https://en.wikipedia.org/wiki/Robot_Scientist}{Adam robot} is an existence
proof that this can be done in a non-trivial way.

\subsection{Starting small}

Maybe begin with experiments that can be done entirely on a computer.

* Number theories, e.g. conjectures about prime numbers
* Solving mazes and other puzzles.
* Mice? Drosophila? Bacteria or yeast in Petri dishes?

I'm afraid that exhausts my imagination. But I think well-designed puzzles could be a
fruitful area, and could give me something tangible to work on until I figure out how to
work on real scientific problems.

Yeast in Petri dishes is how Ross King got started, maybe that's the way to go.

\end{document}
